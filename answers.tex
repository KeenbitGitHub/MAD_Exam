\documentclass{report}
\usepackage[utf8]{inputenc}
\usepackage{tabularx}
\usepackage{listings}
\usepackage{graphicx}
\usepackage{amsmath}
\usepackage{bm}
\usepackage{array}
\usepackage{amssymb}

\title{MAD 2019/2020 Exam Answers}
\author{Exam Number: 18}
\date{January 13, 2020}

\begin{document}

\maketitle

\subsection*{Exercise 1}
I find the task of this exercise a little ambiguous, as I am not sure if I am supposed to (only) compute the mean point of the data, the two eigenvalues and the two eigenvectors or if I am also supposed to convert the dimensionality. To be on the safe side, I will also convert the dimensionality, however, if I was not supposed to do this, please just overlook that part of my answer. \\
\\
We are given the following $N = 8$ 2-dimensional points:
\begin{center}
    \begin{tabular}{|l|l|l|l|l|l|l|l|l|}
        \hline
        coordinate x & 0.1 & 0.5 & 1.1 & -0.5 & 1.3 & 0.2 & -0.1 & 1 \\
        \hline
        coordinate y & 1 & 1 & 2 & 0.2 & -0.1 & -0.1 & -1.5 & -2.5 \\
        \hline
    \end{tabular}
\end{center}
First of we need to make each row have zero mean. This is done, by first computing $\bar{x}$ and $\bar{y}$:
$$\bar{x} = \frac{1}{8} \sum^8 _{n = 1} x_n = 0.45, \quad \quad \bar{y} = \frac{1}{8} \sum^8 _{n = 1} y_n = 0$$

\noindent Which gives us the mean point of data, $(0.45, 0)$. $\bar{x}$ and $\bar{y}$ are then subtracted from respectively coordinate x and coordinate y, which results in the following rows, with zero mean:
\begin{center}
    \begin{tabular}{|l|l|l|l|l|l|l|l|l|}
        \hline
        coordinate x & -0.35 & 0.05 & 0.65 & -0.95 & 0.85 & -0.25 & -0.55 & 0.55 \\
        \hline
        coordinate y & 1 & 1 & 2 & 0.2 & -0.1 & -0.1 & -1.5 & -2.5 \\
        \hline
    \end{tabular}
\end{center}
From this we can extract the following matrix:
\begin{center}
    \begin{math}
        \bm{S} = 
        \begin{bmatrix}
            -0.35 & 1 \\
            0.05 & 1 \\
            0.65 & 2 \\
            -0.95 & 0.2 \\
            0.85 & -0.1 \\
            -0.25 & -0.1 \\
            -0.55 & -1.5 \\
            0.55 & -2.5
        \end{bmatrix}
    \end{math}
\end{center}
Which we use to compute the covariance matric $\bm{C}$:
\begin{center}
    \begin{math}
        \bm{C} = \frac{1}{N} \cdot \bm{S}^T \cdot \bm{S}
        = \frac{1}{8} \cdot 
        \begin{bmatrix}
            -0.35 & 0.05 & 0.65 & -0.95 & 0.85 & -0.25 & -0.55 & 0.55 \\
            1 & 1 & 2 & 0.2 & -0.1 & -0.1 & -1.5 & -2.5
        \end{bmatrix}
        \cdot 
        \begin{bmatrix}
            -0.35 & 1 \\
            0.05 & 1 \\
            0.65 & 2 \\
            -0.95 & 0.2 \\
            0.85 & -0.1 \\
            -0.25 & -0.1 \\
            -0.55 & -1.5 \\
            0.55 & -2.5
        \end{bmatrix} \newline
        =
        \begin{bmatrix}
            0.355 & 0.025 \\
            0.025 & 1.82
        \end{bmatrix}
    \end{math}
\end{center}
Now that we have found the covariance matrix $\bm{C}$, we can find the eigenvalues. First we need to compute $\bm{C}-\lambda \cdot \bm{I}_2$:
\begin{center}
    \begin{math}
        \bm{C} - \lambda \cdot \bm{I}_2 =
        \begin{bmatrix}
            0.355 & 0.025 \\
            0.025 & 1.82
        \end{bmatrix}
        -
        \begin{bmatrix}
            \lambda & 0 \\
            0 & \lambda
        \end{bmatrix}
        =
        \begin{bmatrix}
            0.355 - \lambda & 0.025 \\
            0.025 & 1.82 - \lambda
        \end{bmatrix}
    \end{math}
\end{center}
Which we need to find the determinant of:
\begin{center}
    \begin{math}
        det(\bm{C}-\lambda \cdot \bm{I}_2) = 
        \begin{vmatrix}
            0.355 - \lambda & 0.025 \\
            0.025 & 1.82 - \lambda
        \end{vmatrix} \newline
        =
        (0.355 - \lambda) \cdot (1.82 - \lambda) - 0.025 * 0.025 \newline
        = 0.355 \cdot 1.82 - \lambda \cdot 1.82 + \lambda^2 - 0.025 \cdot 0.025 - 0.355 \cdot \lambda \newline
        \approx \lambda^2 - 2.175 \lambda + 0.6455
    \end{math}
\end{center}
And now solve for $\lambda$:
$$\lambda^2 - 2.175 \lambda + 0.6455 = 0$$
$\Leftrightarrow$
$$\lambda \ \frac{-(-2.175) \pm \sqrt{(-2.175)^2 - 4 \cdot 1 \cdot 0.6455}}{2 \cdot 1} \approx 
\begin{cases}
    1.820 \\
    0.355
\end{cases}$$
Thus, we have found the two eigenvalues $\lambda_1 = 1.820$ and $\lambda_2 = 0.355$. \\

Now, for finding the associated eigenvectors, we first let $\lambda = \lambda_1$ and find $\bm{B} = \bm{C} - \lambda \cdot \bm{I}_2$:
\begin{center}
    \begin{math}
        \bm{B} = \bm{C} - \lambda \cdot \bm{I}_2 =
        \begin{bmatrix}
            0.355 - 1.820 & 0.025 \\
            0.025 & 1.82 - 1.820
        \end{bmatrix}
        \newline =
        \begin{bmatrix}
            -1.4658 & 0.025 \\
            0.025 & 0 \\
        \end{bmatrix}
    \end{math}
\end{center}
Now, we put $\bm{B} \bm{x} = \bm{0}$ on row echelon form
$$\bm{B} \bm{x} = \bm{0}$$
$\Rightarrow$
\begin{center}
    \begin{math}
        \left[
            \begin{array}{cc|c}
                -1.4658 & 0.025 & 0 \\
                0.025 & 0 & 0 \\
            \end{array}
        \right]
    \end{math}
\end{center}
Which is done by adding row 1 times 0.0171 to row 2. Thus, we get the following augmented matrix:
\begin{center}
    \begin{math}
        \left[
            \begin{array}{cc|c}
                -1.4658 & 0.025 & 0 \\
                0 & 0 & 0 \\
            \end{array}
        \right]
    \end{math}
\end{center}
Thus, we have the following equation, where we isolate $x_1$:
$$-1.4658 \cdot x_1 + 0.025 \cdot x_2 = 0$$
$\Leftrightarrow$
$$0.025 \cdot x_2 = 1.4658 \cdot x_1$$
$\Leftrightarrow$
$$0.171 \cdot x_2 = x_1$$
From this point, we can choosing any integer for $x_2$ and get the eigenvector for the eigenvalue $1.820$. I am here choosing $x_2 = 1$:
$$0.171 \cdot 1 = x_1$$
$\Leftrightarrow$
$$0.171 = x_1$$
Thus, the eigenvector for the eigenvalue $1.820$ is:
\begin{center}
    \begin{math}
        \left[
            \begin{array}{c}
                0.171 \\
                1
            \end{array}
        \right]
    \end{math}
\end{center}

Now, we let $\lambda = \lambda_2$ and follow the same procedure:
\begin{center}
    \begin{math}
        \bm{B} = \bm{C} - \lambda \cdot \bm{I}_2 =
        \begin{bmatrix}
            0.355 - 0.355 & 0.025 \\
            0.025 & 1.82 - 0.355
        \end{bmatrix}
        \newline =
        \begin{bmatrix}
            0 & 0.025 \\
            0.025 & 1.465 \\
        \end{bmatrix}
    \end{math}
\end{center}
$$\bm{B} \bm{x} = \bm{0}$$
$\Rightarrow$
\begin{center}
    \begin{math}
        \left[
        \begin{array}{cc|c}
            0 & 0.025 & 0 \\
            0.025 & 1.465 & 0 \\
        \end{array}
        \right]
    \end{math}
\end{center}
To put the matrix on row echelon form, we simply swap row 1 and row 2 and optain the following matrix
\begin{center}
    \begin{math}
        \left[
        \begin{array}{cc|c}
            0.025 & 1.465 & 0 \\
            0 & 0.025 & 0 \\
        \end{array}
        \right]
    \end{math}
\end{center}
Thus, we have the following equation, where $x_1$ needs to be isolated:
$$0.025 \cdot x_1 + 1.465 \cdot x_2 = 0$$
$\Leftrightarrow$
$$1.465 \cdot x_2 = -0.025 \cdot x_1$$
$\Leftrightarrow$
$$ -58.6 \cdot x_2 = x_1$$
Again, here it is possible to choose any value for $x_2$, I will let $x_2 = 1$ and get the following:
$$ -58.6 \cdot 1 = x_1$$
$\Leftrightarrow$
$$ -58.6 = x_1$$
Thus, for the eigenvalue 0.355, we have the eigenvector
\begin{center}
    \begin{math}
        \left[
            \begin{array}{c}
                -58.6 \\
                1
            \end{array}
        \right]
    \end{math}
\end{center}
Thus, sorting the eigenvalues in descending order and the eigenvectors accordingly, we get the following matrix of the eigenvectors:
\begin{center}
    \begin{math}
        \bm{P} =
        \left[
            \begin{array}{cc}
                0.171 & -58.6 \\
                1 & 1
            \end{array}
        \right]
    \end{math}
\end{center}
The dimensionality of the given data can now be converted by multiplying the zero-mean matrix, $\bm{S}$ with $\bm{P}$:
\begin{center}
    \begin{math}
        \begin{bmatrix}
            -0.35 & 1 \\
            0.05 & 1 \\
            0.65 & 2 \\
            -0.95 & 0.2 \\
            0.85 & -0.1 \\
            -0.25 & -0.1 \\
            -0.55 & -1.5 \\
            0.55 & -2.5
        \end{bmatrix}
        \cdot
        \left[
            \begin{array}{cc}
                0.171 & -58.6 \\
                1 & 1
            \end{array}
        \right]
        =
        \begin{bmatrix}
            0.9402 & 21.51 \\
            1.0086 & -1.93 \\
            2.1112 & -36.09 \\
            0.03755 & 55.87 \\
            0.04535 & -49.91 \\
            -0.1428 & 14.55 \\
            -1.5941 & 30.73 \\
            -2.406 & -34.73
        \end{bmatrix}
    \end{math}
\end{center}

\section*{Exercise 2}
First, we compute $Gini(T_{org}, \lambda)$:
$$Gini(T_{org}, \lambda) = \frac{17}{28} \cdot gini(T_{first}) + \frac{11}{28} \cdot gini(T_{second})$$
$$= \frac{17}{28} \left(1 - \left(\frac{6}{17}\right)^2 - \left( \frac{1}{17}^2 \right) - \left( \frac{10}{17} \right)^2 \right) + \frac{11}{28} \left(1 - \left( \frac{1}{11} \right)^2 - \left( \frac{7}{11} \right)^2 - \left( \frac{3}{11} \right)^2 \right)$$
$$ = 0.5206$$

Next, we compute $Gain(T_{orig}, \lambda)$:
$$Gain(T_{orig}, \lambda) = H(7, 8, 13) - \frac{17}{28} \cdot H(6, 1, 10) - \frac{11}{28} \cdot H(1, 7, 3)$$
$$ = \left(-\frac{13}{28} \cdot \log_2 \left(\frac{13}{28} \right) - \frac{8}{28} \cdot \log_2 \left( \frac{8}{28} \right) - \frac{7}{28} \cdot \left( \frac{7}{28} \right) \right)$$
$$ - \frac{17}{28} \left( - \frac{10}{17} \cdot \log_2 \left( \frac{10}{17} \right) - \frac{6}{17} \cdot \log_2 \left( \frac{6}{17} \right) - \frac{1}{17} \cdot \left( \frac{1}{17} \right) \right)$$
$$ - \frac{11}{28} \left( - \frac{7}{11} \cdot \log_2 \left( \frac{7}{11} \right) - \frac{3}{11} \cdot \log_2 \left( \frac{3}{11} \right) - \frac{1}{11} \cdot \log_2 \left( \frac{1}{11} \right) \right)$$
$$= 0.3015$$

Thus, it has been computed, that $Gini(T_{org}, \lambda) = 0.5206$ and $Gain(T_{orig}, \lambda) = 0.3015$
\end{document}